\chapter{Introducción}

Los problemas de optimización multiobjetivo \textbf{(POM)} ha sido de gran utilidad en la comunidad científica e industrial, ya que son comunes aquellos problemas que requieren
considerar múltiples objetivos, incluyendo tiempo, costos, cantidad, calidad, etc. Estos objetivos se encuentran típicamente en conflicto y se pueden encontrar inmersos en
diferentes problemas como: ruteo de vehículos, localización, cadenas de suministro, calendarización, entre otros \cite{coello2004applications}.\\

Esta clase de problemas ha sido tradicionalmente atacada reduciendo el \textbf{POM} a un problema mono-objetivo (estrategias de suma ponderada, de restricciones-$\epsilon$,
de programación por metas). Desde los años 1990, los esfuerzos de investigación se han dedicado al desarrollo de técnicas metaheurísticas y, particularmente, Algoritmos Evolutivos (AEs).
Desde el punto de vista metaheurístico, se han generado tres grandes enfoques para resolver el \textbf{POM}:

\begin{itemize}
 \item Basado en descomposición.
 \item Basado en dominancia.
 \item Basado en indicadores.
\end{itemize}

El enfoque de descomposición explícitamente descompone un problema multiobjetivo en $N$ problemas de optimización escalar y suele introducir técnicas poblacionales para
la resolución simultánea de ellos. Bajo este enfoque, es importante seleccionar los problemas escalares conforme a algún criterio que permita mejorar la diversidad del conjunto
de soluciones arrojado, dejando la convergencia al submétodo de resolución como podría ser algoritmos geneticos, estrategias evolutivas, etc. Un marco de trabajo popular con este enfoque es el algoritmo evolutivo multiobjetivo basado en
descomposición (\emph{MOEA/D}) \cite{4358754}.
\newline

En el enfoque de dominancia, se utiliza la relación de dominancia para inducir un orden parcial en el espacio de los objetivos y, de esta manera, decidir cuándo una solución
en dicho espacio es comparable con otra y, en este caso, cuál es la mejor. Cabe mencionar que dicho orden parcial está limitado en espacios de altas dimensiones puesto que una gran
cantidad de soluciones serán no comparables, mermando en consecuencia la efectividad de los procedimientos de selección. El algoritmo genético de ordenamiento no dominado
(\emph{NSGA-III}) \cite{6600851} es un marco de trabajo popular con esta estrategia, que incorpora además técnicas de nichos para mantener la diversidad.
\newline

Finalmente, el enfoque basado en indicadores convierte el problema multiobjetivo original en un problema mono o multiobjetivo diferente, que optimiza el valor de algún indicador de desempeño,
representando la calidad del conjunto de soluciones obtenidas tanto en términos de convergencia como de dispersión y uniformidad de su distribución. Un indicador popular
es el hipervolumen ya que es la único acorde a la definición de  optimalidad de Pareto (i.e., las soluciones Pareto-óptimas son aquellas que maximizan el hipervolumen).
Sin embargo, calcular el hipervolumen se vuelve costoso conforme aumenta la dimensión del problema multiobjetivo original. Por ejemplo, la mejor forma conocida de calcular
el hipervolumen tiene una complejidad de $O(n^2m^3)$ \cite{FonPaqLop06:hypervolume}, donde $n$ es el número de soluciones aproximadas y $m$ el número de objetivos.
\newline

Un conjunto de soluciones candidatas entregadas por cualquiera de los enfoques antes mencionados debe de cumplir con criterios de convergencia y diversidad para ser
considerado aceptable. Esta meta de obtener un conjunto de soluciones diversas y uniformemente distribuidas se ve dificultada cuando el \textbf{POM} tiene alguna de las siguientes características:

\begin{itemize}
 \item Convexidad o no convexidad del frente óptimo de Pareto.
 \item Discontinuidad del frente óptimo de Pareto.
 \item Densidad no uniforme de las soluciones en el frente óptimo de Pareto.
 \item Degeneración del frente óptimo de Pareto.
\end{itemize}

En general, los enfoques basados en dominancia y descomposición utilizan puntos y/o vectores de referencia, que representan direcciones de búsqueda en el espacio objetivo. Si bien se han estudiado distintos métodos para generarlos y que se detallan posteriormente,
sus valores se mantienen clásicamente fijos durante el transcurso de la ejecución del algoritmo. Esta estrategia de puntos/vectores de referencia estáticos experimenta problemas cuando el frente óptimo presenta las fuentes de dificultad antes mencionadas.\\

Las hipótesis centrales del trabajo son las siguientes:

\begin{itemize}
 \item La técnica de inicialización de los puntos de referencia/vectores de pesos tiene un impacto directo en el desempeño de los algoritmos evolutivos multiobjetivo.
 \item Es posible actualizar los puntos de referencia/vectores de pesos en el transcurso  de la busqueda mediante la estrategia de repulsión de subpoblaciones.
\end{itemize}

\section{Objetivo general}

Adaptar e integrar estrategias de generación y actualización de los vectores de pesos en el MOEA/D y de los puntos de referencia en el NSGA-III, para  evaluar su impacto sobre el desempeño en términos de diversidad de estos dos algoritmos evolutivos multiobjetivo.

\section{Objetivos particulares}

\begin{itemize}

\item Adaptar, implementar y determinar el comportamiento del método \emph{MOEA/D} para las instancias seleccionadas \cite{zhang2008multiobjective}.
 
\item Adaptar, implementar y determinar el comportamiento del método \emph{NSGA-III} para las instancias seleccionadas \cite{zhang2008multiobjective}.

\item Generar vectores de pesos/puntos de referencia iniciales para \emph{MOEA/D} y \emph{NSGA-III}.

\item Desarrollar una estrategia de actualización de vectores de pesos para el método \emph{MOEA/D}, basada en el método de repulsión de subpoblaciones \cite{ahrari2016multimodal}.

\item Desarrollar una estrategia de actualización de puntos de referencia para el método \emph{NSGA-III}, basada en el método de repulsión de subpoblaciones \cite{ahrari2016multimodal}.
  
%\item Realizar experimentos computacionales sobre un banco de instancias clásicas de optimización multiobjetivo \cite{zhang2008multiobjective} para comparar las diferentes versiones de los algoritmos estudiados, particularmente con respecto a los indicadores de diversidad.

%\item Realizar experimentos computacionales sobre un banco de instancias clásicas de optimización multiobjetivo \cite{zhang2008multiobjective} para comparar los mejores resultados obtenidos con los reportados en la literatura especializada.  

\end{itemize}

